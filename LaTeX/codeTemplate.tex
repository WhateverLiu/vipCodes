\documentclass{article}
\usepackage{setspace}\onehalfspacing
%\documentclass[journal]{IEEEtran}
\usepackage[title]{appendix}
\usepackage[dvipsnames]{xcolor}
\usepackage{graphicx}
\usepackage{amsmath}
\usepackage{amssymb}
\usepackage{wrapfig}
\usepackage{mdframed}
\usepackage{amsbsy}
\usepackage{hyperref}
\usepackage{wrapfig}
\usepackage{animate}
\usepackage{algpseudocode}
\usepackage{algorithm}
\usepackage{algcompatible}
\usepackage{tcolorbox}
%\usepackage{caption}
\usepackage[font={small,sf}]{caption}
\usepackage{algorithmicx}
\usepackage{algpseudocode}
\usepackage{float}
\usepackage{multicol}
\usepackage{caption}
\usepackage{amsfonts}
\usepackage{subfig}
\usepackage{color}
\usepackage{subfig}
%\usepackage[demo]{graphicx}
\usepackage{url}
\usepackage[export]{adjustbox}[2011/08/13]
\usepackage{textpos}
\usepackage[percent]{overpic}


\DeclareMathOperator*{\argmax}{arg\,max}
\DeclareMathOperator*{\argmin}{arg\,min}


%\newcommand{\minus}{\scalebox{0.8}{$-$}}
\newcommand{\conv}{\scalebox{0.6}{$\perp$}}
\newcommand{\como}{\scalebox{0.6}{$+$}}


\definecolor{mygreen}{rgb}{0,0.45,0}
\definecolor{mygray}{rgb}{0.99,0.99,0.99}
\definecolor{mymauve}{rgb}{0.58,0,0.82}
\definecolor{darkblue}{rgb}{0.0, 0.0, 0.55}


\usepackage{listings,lstautogobble}
\lstset{
	autogobble=true,
	xleftmargin=0.25pt,
	backgroundcolor=\color{mygray},   % choose the background color; you must add \usepackage{color} or \usepackage{xcolor}; should come as last argument
	basicstyle=\small\ttfamily,        % the size of the fonts that are used for the code
	breakatwhitespace=false,         % sets if automatic breaks should only happen at whitespace
	breaklines=true,                 % sets automatic line breaking
	captionpos=t,                    % sets the caption-position to bottom
	commentstyle=\color{mygreen},    % comment style
	%deletekeywords={...},            % if you want to delete keywords from the given language
	escapeinside={\%*}{*)},          % if you want to add LaTeX within your code
	extendedchars=true,              % lets you use non-ASCII characters; for 8-bits encodings only, does not work with UTF-8
%	frame=single,	                   % adds a frame around the code
	keepspaces=true,                 % keeps spaces in text, useful for keeping indentation of code (possibly needs columns=flexible)
	keywordstyle=\color{blue},       % keyword style
	language=R,                 % the language of the code
	morekeywords={*,...},           % if you want to add more keywords to the set
	numbers=none,                    % where to put the line-numbers; possible values are (none, left, right)
%	numbersep=2pt,                   % how far the line-numbers are from the code
	numberstyle=\tiny\color{white}, % the style that is used for the line-numbers
	rulecolor=\color{black},         % if not set, the frame-color may be changed on line-breaks within not-black text (e.g. comments (green here))
	showspaces=false,                % show spaces everywhere adding particular underscores; it overrides 'showstringspaces'
	showstringspaces=false,          % underline spaces within strings only
	showtabs=false,                  % show tabs within strings adding particular underscores
	stepnumber=1,                    % the step between two line-numbers. If it's 1, each line will be numbered
	stringstyle=\color{mymauve},     % string literal style
	tabsize=2,	                   % sets default tabsize to 2 spaces
	title=\lstname                   % show the filename of files included with \lstinputlisting; also try caption instead of title
	keepspaces=true
	otherkeywords={!,!=,~,$,*,\&,+,-,^,\%,\%/\%,\%*\%,\%\%,<-,<<-,_,/}
}


\usepackage{geometry}
\geometry{a4paper, left=1.25in, top=1.25in, right=1.25in, bottom=1.25in}


%\usepackage[utf8]{inputenc}
%\usepackage{times}

%\renewcommand{\familydefault}{cmr}



\hypersetup{
	colorlinks   = true,
	linkcolor    = blue,
	urlcolor = blue
}

\begin{document}

\title{The title}
\author{Charlie Wusuo Liu}
\date{June 1, 2021}


	\maketitle
	\begin{abstract}
       
	\end{abstract}
	\tableofcontents	

\pagestyle{plain}


\section{Problem 1 (Buying/selling stocks)}




%\clearpage
%
%\section{Code section}
%
%\renewcommand{\lstlistingname}{Code}
%\begin{lstlisting}[caption={Fractional factorial design.}, label={ffd}]
%tmp = FrF2::FrF2(16, nfactors = 10)
%rst = as.data.frame(tmp)
%cn = colnames(rst)
%rst = rst[colnames(rst)]
%rst = as.data.frame(matrix(as.integer(as.matrix(rst)), ncol = 10))
%colnames(rst) = cn
%write.csv(rst, file = "data/frf2.csv", row.names = F)
%tmp = t(rst) % * % as.matrix(rst)
%write.csv(tmp, file = "data/innerProd.csv")
%\end{lstlisting}
%
%\clearpage
%
%\renewcommand{\lstlistingname}{Code}
%\begin{lstlisting}[caption={Simulation R code. The code requires R package "Rcpp" and Rtools35.}, label={subsetreg}]
%Rcpp::sourceCpp("cppcode.cpp", verbose = T)
%arriveLambda = 50
%Npeople = 1e6L
%
%
%combo = as.data.frame(t(expand.grid(Nserver = 31:50, Nscanner = 31:50)))
%mt = list()
%for(i in 1:length(combo))
%{
%	cat(i, "")
%	Nserver = combo[[i]][1]
%	Nscanner = combo[[i]][2]
%	
%	
%	singleServerMean = 0.75
%	scannerMinMax = c(0.5, 1)
%	allServerMean = singleServerMean / Nserver
%	
%	
%	tArrival = cumsum(rexp(Npeople, rate = arriveLambda))
%	dtIDcheck = rexp(Npeople, rate = 1 / allServerMean)
%	dtScan = runif(Npeople, scannerMinMax[1], scannerMinMax[2])
%	
%	
%	tAfterIDcheck = tIDcheckEnd(tArrival, dtIDcheck)
%	tAfterScan = tScanEnd(tAfterIDcheck, Nscanner, dtScan)
%	dt = tAfterScan - tArrival
%	mt[[i]] = mean(dt)
%}
%
%
%tmp = data.frame(expand.grid(Nserver = 31:50, Nscanner = 31:50), t = unlist(mt))
%tmp = tmp[tmp$t < 15,]
%tmp = tmp[order(tmp$Nserver + tmp$Nscanner), ]
%tmp
%\end{lstlisting}
%
%
%\clearpage
%
%\renewcommand{\lstlistingname}{Code}
%\begin{lstlisting}[caption={"cppcode.cpp": Simulation C++ functions.}, label={len}]
%// [[Rcpp::plugins(cpp11)]]
%# include <Rcpp.h>
%using namespace Rcpp;
%# define vec std::vector
%
%
%// [[Rcpp::export]]
%NumericVector tIDcheckEnd(NumericVector tArrival, NumericVector dtIDcheck)
%{
%	NumericVector rst(tArrival.size());
%	rst[0] = tArrival[0] + dtIDcheck[0];
%	for(int i = 1, iend = tArrival.size(); i < iend; ++i)
%	{
%		rst[i] = std::max<double> (
%		rst[i - 1], tArrival[i]) + dtIDcheck[i];
%	}
%	return rst;
%}
%
%
%struct it { int id; double time; };
%struct fastQueue
%{
%	vec<it> t;
%	it *head, *tail;
%	void set(int capacity)
%	{
%		t.resize(capacity);
%		head = &t[0]; 
%		tail = head;
%	}
%	fastQueue() {}
%	fastQueue(int capacity) { set(capacity); }
%	void push(int id, double time) 
%	{ 
%		++tail; 
%		(tail - 1)->id = id;
%		(tail - 1)->time = time;
%	}
%	int size() { return tail - head; }
%	void pop(double currentTime,
%	double *dtScan,
%	double *endTime)
%	{
%		while(head < tail)
%		{
%			double tmpt = head->time + dtScan[head->id];
%			if(tmpt >= currentTime) break;
%			endTime[head->id] = tmpt;
%			++head;
%		}
%	}
%};
%
%
%int shortestQ(vec<fastQueue> &Q)
%{
%	int rst = 0, leastSize = Q[0].size();
%	for(int i = 1, iend = Q.size(); i < iend; ++i)
%	{
%		if(Q[i].size() < leastSize)
%		{
%			leastSize = Q[i].size();
%			rst = i;
%		}
%	}
%	return rst;
%}
%
%
%// timeAfterIDcheck is sorted.
%// [[Rcpp::export]]
%NumericVector tScanEnd(NumericVector timeAfterIDcheck,
%int Nscanner, NumericVector dtScan)
%{
%	vec<fastQueue> Q(Nscanner);
%	for(int i = 0; i < Nscanner; ++i) Q[i].set(dtScan.size());
%	NumericVector tend(dtScan.size());
%	for(int i = 0, iend = dtScan.size(); i < iend; ++i)
%	{
%		for(int j = 0; j < Nscanner; ++j)
%		{
%			Q[j].pop(timeAfterIDcheck[i], &dtScan[0], &tend[0]);
%		}
%		int q = shortestQ(Q);
%		Q[q].push(i, timeAfterIDcheck[i]);
%	}
%	for(int j = 0; j < Nscanner; ++j)
%	{
%		Q[j].pop(1e300, &dtScan[0], &tend[0]);
%	}
%	return tend;
%}
%\end{lstlisting}


\end{document}

































